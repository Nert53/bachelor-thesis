%%%  Šablona práce od:
%%%  Copyright (C) 2012 Martin Rotter, <rotter.martinos@gmail.com>
%%%  Copyright (C) 2014 Jan Outrata, <jan.outrata@upol.cz>

%%  Pro získání PDF souboru dokumentu je třeba tento zdrojový text v
%%  LaTeXu přeložit (dvakrát) programem pdfLaTeX.

%%  V případě použití programu BibLaTeX pro tvorbu seznamu literatury
%%  je poté ještě třeba spustit program Biber s parametrem jméno
%%  souboru zdrojového textu bez přípony a následně opět (dvakrát)
%%  přeložit zdrojový text programem pdfLaTeX.

%%  Postup získání Postscriptového souboru je popsán v dokumentaci.

%%  Třída dokumentu implementující styl pro závěrečnou práci. Vybrané
%%  nepovinné parametry (ostatní v dokumentaci):

%%  'master' pro sazbu diplomové práce, jinak se sází bakalářská práce

%%  'program=kód' pro Váš studijní program/obor (specializaci), kódy
%%  pro diplomovou práci 'infoi' pro Informatiku (Obecná informatika),
%%  'infui' pro Informatiku (Umělá inteligence), 'ainfpst' pro
%%  Aplikovanou informatiku (Počítačové systémy a technologie), 'uinf'
%%  pro Učitelství informatiky pro střední školy, 'binf' pro
%%  Bioinformatiku, 'inf' pro Informatiku (bez specializací) a 'ainf'
%%  pro Aplikovanou informatiku (bez specializací), jinak je výchozí
%%  ainfvs pro Aplikovanou informatiku (Vývoj software), a pro
%%  bakalářskou práci 'infoi' pro Informatiku (Obecná informatika),
%%  'itp' pro Informační technologie v prezenční formě, 'itk' pro
%%  Informační technologie v kombinované formě, 'infv' pro Informatiku
%%  pro vzdělávání, 'binf' pro Bioinfomatiku, 'inf' pro Informatiku
%%  (bez specializací), 'ainfp' pro Aplikovanou informatiku (bez
%%  specializací) v prezenční formě, 'ainfk' pro Aplikovanou
%%  informatiku (bez specializací) v kombinované formě, jinak je
%%  výchozí infpvs pro Informatiku (Programování a vývoj software)

%%  'printversion' pro sazbu verze pro tisk (nebarevné logo a odkazy,
%%  odkazy s uvedením adresy za odkazem, ne odkazy do rejstříku),
%%  jinak verze pro prohlížeč

%%  'biblatex' pro zapnutí podpory pro sazbu bibliografie pomocí
%%  BibLaTeXu, jinak je výchozí sazba v prostředí thebibliography

%%  'language=jazyk' pro jazyk práce, jazyky english pro anglický,
%%  slovak pro slovenský, jinak je výchozí czech pro český

%%  'font=sans' pro bezpatkový font (Iwona Light), jinak je výchozí
%%  serif pro patkový (Latin Modern)

%%  'figures, tables, theorems a sourcecodes' pro sazbu seznamu
%%  obrázků, tabulek, vět a zdrojových kódů, jinak při =false se
%%  nesází (u theorems a sourcecodes výchozí)

\documentclass[
%  master,
%  program=ainfvs,
%  printversion,
  biblatex,
%  language=english,
%  font=sans,
  figures=false,
  tables=false,
%  theorems,
%  sourcecodes,
  glossaries,
  index
]{kidiplom}

%% Informace pro úvodní strany. V jazyku práce (pokud není v komentáři
%% uvedeno česky) a anglicky. Uveďte všechny, u kterých není v
%% komentáři uvedeno, že jsou volitelné. Při neuvedení se použijí
%% výchozí texty. Text pro jiný než nastavený jazyk práce (nepovinným
%% parametrem language makra \documentclass, výchozí český) se zadává
%% použitím makra s uvedením jazyka jako nepovinného parametru.

%% Název práce, česky a anglicky. Měl by se vysázet na jeden řádek.
\title{Multiplatformní aplikace pro správu osobních financí}
\title[english]{Cross-platform application for personal finance management}

%% Jméno autora práce. Makro nemá nepovinný parametr pro uvedení
%% jazyka.
\author{Vojtěch Netrh}

%% Jméno vedoucího práce (včetně titulů). Makro nemá nepovinný
%% parametr pro uvedení jazyka.
\supervisor{doc. RNDr. Jan Konečný, Ph.D.}

%% Volitelný rok odevzdání práce. Výchozí je aktuální (kalendářní)
%% rok. Makro nemá nepovinný parametr pro uvedení jazyka.
\yearofsubmit{2025}

%% Anotace práce, včetně anglické (obvykle překlad z jazyka
%% práce). Jeden odstavec!
\annotation{Ukázkový text závěrečné práce na Katedře informatiky
  Přírodovědecké fakulty Univerzity Palackého v Olomouci, který je
  zároveň dokumentací stylu pro text práce v \LaTeX{}u. Zdrojový text
  v \LaTeX{}u je doporučeno použít jako šablonu pro text skutečné
  závěrečné práce studenta.}

\annotation[english]{Sample text of thesis at the \kitextdepten,
  \kitextfacultyen, \kitextuniven{} and, at the same time,
  documentation of the \LaTeX{} style for the text. The source text in
  \LaTeX{} is recommended to be used as a template for real student's
  thesis text.}

%% Klíčová slova práce, včetně anglických. Oddělená středníkem.
\keywords{Flutter; Dart; multiplatformní; osobní finance}
\keywords[english]{Flutter; Dart; cross-platform; personal finance}

%% Volitelná specifikace příloh textu práce, i anglicky. Výchozí je
%% 'elektronická data v systému katedry informatiky / electronic data
%% in system of department of computer science'.
%\supplements{nejlepší software všech dob}
%\supplements[english]{the best software of all times}

%% Volitelné poděkování. Stručné! Výchozí je prázdné. Makro nemá
%% nepovinný parametr pro uvedení jazyka.
\thanks{Děkuji panu doc. RNDr. Janu Konečnému Ph. D. za cenné rady a podněty při tvorbě práce. Svým blízkým za podporu běheme celého bakalářského studia.}

%% Cesta k souboru s bibliografií pro její sazbu pomocí BibLaTeXu
%% (zvolenou nepovinným parametrem biblatex makra
%% \documentclass). Použijte pouze při této sazbě, ne při (výchozí)
%% sazbě v prostředí thebibliography.
\bibliography{bibliografie.bib}

%% Další dodatečné styly (balíky) potřebné pro sazbu vlastního textu
%% práce.
\usepackage{lipsum}
\usepackage{longtable}

\begin{document}
%% Sazba úvodních stran -- titulní, s bibliografickými údaji, s
%% anotací a klíčovými slovy, s poděkováním a prohlášením, s obsahem a
%% se seznamy obrázků, tabulek, vět a zdrojových kódů (pokud jejich
%% sazba není vypnutá).
\maketitle

%% Vlastní text závěrečné práce. Pro povinné závěry, před přílohami,
%% použijte prostředí kiconclusions. Povinná je i příloha s obsahem
%% elektronických dat.

%% -------------------------------------------------------------------

\newcommand{\BibLaTeX}{\textsc{Bib}\LaTeX}

% ----- tady začíná MNOU PSANÝ TEXT

\section{Úvod}
Peníze se vyskytují všude kolem nás a chceme-li nebo ne, hrají v našich životech podstatnou roli. Z hlediska osobního pohledu jednoho člověka je užitečné mít ve svých financích pořádek. Můžeme tak ušetřit peníze, případně je efektivněji využívat. V neposlední řadě by měl člověk vědět kolik peněz by potřeboval v případě výpadku příjmů a jaká má být jeho \uv{železná rezerva}.

V dnešní době již většina lidí používá internetové bankovnictví, která dost často poskytuje alespoň základní statistiky o našem nakládání s financemi. Na druhou stranu lidé mívají účty u více bank a reporty o stavu financí nejsou dostatečně přizpůsobitelné.

Kromě toho, že peníze se podaří člověku ušetřit je vhodné s nimi potom dále nakládat. Můžeme je ihned utratit (což není dlouhodobě vhodná varianta), spořit si nebo dále investovat. Spořit si je možné na řadu věcí - nové bydlení, automobil či vytvoření dostatečného objemu pěnez pro založení vlastního podnikání. Pro spoření existují dva základní způsoby jak peníze ukládat. Prvním z nich je mít je na běžném či spořícím účtu a jejich hodnota bude v čas pořád zhruba stejná. Druhým přístupem je aktivně investovat a snažit se pomocí nich vydělat další peníze. V dnešní době internetu a široké osvěty v této oblasti nabýcá druhý způsob stále více na popularitě.

\subsection{Motivace}
Požadované funkcionality na tuto aplikaci se současně odvíjely ode mě i mého vedoucího pane Konečného. Oba jsme si představovali mít jednoduchou aplikaci pro správu našich finančních prostředků. Já jsem některé aplikace před začátkem tvorby této práce používal, ale velmi dlouhou dobu mi trvalo najít nějakou, se kterou bych byl spokojen tak, abych ji byl ochoten používat na denní bázi, ikdyž s kompromisy.

\subsection{Požadavky}
Na základě mé zkušenosti s již existujícími aplikacemi a počátečních představ pana Konečného jsem si zvolil následující body ke splnění:

\begin{itemize}
  \item \textbf{Jednoduchost více než komplexní funkce} - snažit se implementovat dostatečný počet funkcí, avšak nezahltit uživatele příliš mnoha různými nastaveními, které mu přidají práci při volení parametrů. Pokud je pro uživatele pohodlné zapisovat transakce již v průběhu dne, aniž by měl pocit, že u aplikace tráví moc času, je to ideální scénář.
  \item \textbf{Mobilní telefony i počítače} - obsáhnout zařízení na různé škále velikosti přináší dostupnost aplikace pro více uživatelů. Někdo rád evidenci financí dělá na denní bázi (obvykle pomocí mobilního telefonu), někdo naopak až měsíc zpětně. Na zpracování více dat najednou je jistě počítač s větším displejem vhodnější volbou.
  \item \textbf{Zahraniční měny} - jelikož poměrně často cestuji a prakticky využívám v zahraničí pouze platby kartou, byla pro mě toto jasná podmínka. Kurz měny byl měl ideálně jít libovolně upravit, případně automaticky synchronizovat z internetu.
  \item \textbf{Export dat} - uživatel by měl mít možnost exportovat data v běžně používaných formátech jako je CSV nebo JSON. Export považuji za důležitý z důvodu přenesení dat do jiného prostředí či aplikace.
\end{itemize}

\section{Přehled existujících řešení}
K danému tématu již exstuje řada aplikací nabízejících nástroje pro správu financí. Každé řešení přistupuje k problému jiným způsobem. 

Nejblíže je z hlediska uživatele poskytnutí základních nástrojů přímo v bankovní aplikaci. Z mého pohledu je to nejméně vhodné řešení hned z několika důvodů:
\begin{itemize}
  \item obvykle má člověk účet u více bank (aplikace jedné z nich neposkytuje přehled o celkových financích),
  \item jen část (i když v dnešní době většinová) transakcí je prováděna pomocí platební karty,
  \item uživatel nemusí chtít mít všechny pohyby na účtu zahrnuty do celkové analýzy.
\end{itemize}

Aplikace třetích stran v tomto segmentu, nabízí různé možnosti. Mobilní aplikace bývají obvykle jednodušší a s přívětivějším uživatelským rozhraním. Zatímco desktopové aplikace mají uživatelské rozhraní spíše starší, avšak poskytují nepřebernou škálu funkcí.

\subsection{Wallet}
Mobilní aplikace dostupné na platformy Android i iOS \cite{wallet}, za jejíž stažení uživatel v první fázi nezaplatí. Nabízí věechny základní funkce co by uživatel očekával - kategorie transakcí, podpora více měn, poznámky. Z hlediska analýzu a grafového zobrazení je na výběr jen velmi málo možností v z mého pohledu nepřehledném uspořádání. Ze zajímavých funkcionalit stojí za zmínku plánované transakce nebo automatická bankovní synchronizace (tato funkce je už placená).

\subsection{1Money}
Ještě na konci roku 2024 mobilní aplikace dostupná na obě platformy, o pár měsícu později už pouze na iOS. Z mnou vyzkoušených aplikací největší favorit pro denní užití. Aplikace mě zaujala svojí obrazovkou \textit{Categories}.  Středem obrazovky je prstencový graf zobrazující poměr utracených peněz podle kategorií a v jeho středu se ještě nachází dva údaje - stav příjmů a výdajů. Zbytek obrazovky pokrývají kolečka označující kategorie, po jejichž stisknutí je uživateli ihned umožněno přidat transakci s danou kategorií a aktuálním časovým razítkem. Velmi intuitivní a rychlé řešení pro každodenní použití. Umožňuje vytvořit více různých účtů, včetně účtů pro spoření. Z hlediska analýzy a grafů zde najdeme pouze dvě velmi omezené možnosti - již zmíněný prstencový graf kategorií a sloupcový graf znázornění transakcí v čase. Tuto aplikaci jsem si já osobně oblíbil nejvíce.

\subsection{Cashew}
Mobilní aplikace založená na stejné technologii jako moje aplikace - Flutter \cite{cashew}. Implementuje komponenty z Material Design, ale rozvržení obrazovky si autoři uspořádali podle sebe. Uživateli umožňuje široké možnosti nastavení z hlediska vzhledu (barvy, typ písma, typ ikon či formáty data i čísel). Umožňuje zálohování dat na Google disk i import z csv souborů. Podporuje funkci více účtů. Obsahuje i placenou verzi, které je primárně podporu pro vývojáře.

\subsection{Money Manager Ex}
Aplikace dostupná jak na mobilní zařízení (nízký počet stažení \cite{money-manager-android}) tak především i na dekstopové zařízení \cite{money-manager}. Uživatelské rozhraní je na první pohled relativně přívětivé, ikdyž místy obsahuje širokou škálu tlačítek a možností k nastavení. Při prvním spuštění nevyžaduje žádné složité nastavení. Zde se mi velmi líbí velká nabídka grafů k analýze údajů a možnost exportu do PDF pro většinu obrazovek.

\subsection{Homebank}
Desktopová aplikace dostupná na Windows, Linux i macOS \cite{homebank}. Z hlediska uživatelského komfortu je oproti Money Manager Ex jednodušší na používání. Při prvotním nastavení sice člověk musí vhodně nakonfigurovat svůj účet, v dalších fázích je už používání intuitivní. Centrem aplikace je dashboard se stavy účtu, přehledem transakcí a jednoduchými grafy. Všechny ostatní funkce se pak otevírají jako nové okno. Uživteslké rozhraní působí moderním a intuitivním dojmem.

\section{Použité technologie}

Požadavkem byla multiplatformní vývoj existuje řada technologií \cite{jetbrains-crossplatform}, které však k tvorbě aplikace přistupují odlišným způsobem. Mezi nejpopulárnější technologie vhodné na tuto práci se řadí \textit{React Native} a \textit{Flutter}. Jelikož jsem za dobu, co se věnuji programování nepřilnul k webovým technologiím, na kterých primárně stojí React Native, zvolil jsem Flutter. Přispěl k tomu fakt, že se více blíží klasickým objektově orientovaným jazykům a jednoduše implementuje \textit{Material Design}, který rád v uživatelském rozhraní vídám.

Aplikace je otestována a řádně funguje na obou mobilních platformách (Android iOS) a jako desktopová Windows aplikace. Ze stejného kódu jde sestavit i aplikace na macOS a Linux. Tyto platformy jsem však netestoval a nemohu zaručit jejich bezproblémový chod.

\subsection{Dart}
Objektově orientovaný jazyk pohcázející z společnosti Google \cite{dart}, jehož první verze byla zveřejněna 14. listopadu 2013. Momentálně je jeho aktjuální verze 3. Jde o typově bezpečný jazyk, podporující třídy se syntaxí založenou na jazyku C. Může být kompilován do strojového kódu, JavaScriptu nebo WebAssembly. Taktéž je základem frameworku Flutter.

Je dodáván se širokou škálou základních knihoven, které umožňují obstarat běžný vývoj. Jako příklad mohu uvést knihovnu \verb|dart:async|, která umožňuje asynchorní programování za využítí tříd jako je \verb|Future| nebo knihovna \verb|dart:io| pro práci se souborovým systémem či HTTP protokolem.

Technologie jazyka Dart umožňují spustit kód dvěma způsoby \cite{dart}. První z nich je určen pro mobilní a desktopové aplikace. Dart obsahuje virtual machine s JIT\footnote{Just in time} kompilací, který se používá především ve fázi vývoje. Tento způsob umožňuje tzv. inkrementální rekompilaci jejiž výhodou je funkcionalita \textit{hot reload} či nástroje \textit{DevTools} pro aktuální metriky ladění. Dále i AOT\footnote{Ahead of time} kompilátor pro tvorbu strojového kódu, který se využívá při sestavování apliakce pro použití v produkci. Druhý způsob se týká webové platformy, kde Dart umožňuje kód přeložit do JavaScript nebo WebAssembly. Opět i zde Dart využívá různé techniky pro ladění kódu a následné produkci. Oba tyto způsoby jsou stejné v tom že potřebují \textit{běhové prostředí Dart}. Toto prostředí se stará o správu paměti pomocí garbage collector, vynucení kontroly typů a správu vláken.

\begin{kicode}{}{}{Ukázka kódu v jazyce Dart}
class Point {
  final double x;
  final double y;

  const Point(this.x, this.y);

  bool get isInsideUnitCircle => x * x + y * y <= 1;
}
\end{kicode}

\subsection{Flutter}
Open-source framework pro tvorbu uživatelských rozhraní založený společností Google \cite{flutter}. Poprvé zveřejněn byl v květnu 2017. Google sám ho využívá v aplikacích jako je Google Play nebo Goole Earth \cite{flutter-showcase}. Mezi nejznámejší aplikace třetích stran používajích Flutter patří Alibaba nebo hra PUBG Mobile \cite{flutter-showcase}.

Používá své vlastní vykreslovací jádro \cite{flutter-architecture}, která pixely přímo vykresulje na obrazovku. Toto je podstatný rozdíl oproti řadě jiných frameworků, které se spoléhají na vykreslovací jádro dané platformy. Tento přístup umožňje mít identicky vypadající uživatelské rozhraní napříč všemi podporovanými platformami.

Základní komponentou je \textit{widget}. Ten se dále může skládat z dalších widgetů. Celé uživatelksé rozhraní je poté poskládáno z těchto celků. Flutter sám o sobě poskytuje dva typy předdefinovancýh widgetů - Material Design widgety a Cupertio widgety. Přestože oba mají svoji priámární platformu, Flutter je umožňuje používat libovolně kdekoliv. Programátor si samozřejmě sám může definovat widgety vlastní.

Pro rozložení prvků na stránce (vytvoření \textit{layout}) se tatkéž používají widgety, přestože nejsou při zobrazení viditelné. Základními widgety pro tvorbu layoutu jsou \verb|Row|, \verb|Column| a \verb|Container|. Pomocí \verb|Container| můžeme ostatním widgetům přidávat odsazení atd. Tyto widgety slouží pro specifickou či nízkoúrovňovou tvorbu layoutu. Můžeme využít i specializované widgety jako \verb|GridView| pro tvorbu mřížky nebo \verb|ListView| pro rolovatelný seznam. Nejvíce specifické widgety jako \verb|BottomNavigationBar| či \verb|AppBar| zajistí dodržení pravidel stanovaných Material Designem a umožní programátorovi velmi jednoduchou implementaci.

Tohle je test citace \cite{flutter}

\subsection{Material Design}
V březnu 2025 je verze 3 nejnovějším open source designový systém od společnosti Google \cite{m3}. Systémem v tomto případě mám na mysli soubor pravidel pro uživatelské rozhraní, konkrétní komponenty i barevné provedení. Všechny tyto prvky jsou vytvořeny zkušenými designéry s respektem pro psychologický vliv uživatelského rozhraní na člověka.

Předchozí označení verze 3 napovídá, že v minulosti již proběhly aktualizace tohoto systému. Tento krok dává smysl vzhledem k vyvíjejícím se trendům v oblasti designu. Tyto aktualizace mohl běžných uživatel pozorovat v operačním systému Android nebo v aplikacích od Googlu, jelikož Google tento systém používá téměr všude.

Použití tohoto systému má ve Flutteru řadu výhod. Především většina Material widgetů je již od tvůrců implementována \cite{m3-components} a použití programátorem je velmi jednoduché. Programátor nemusí často řešit rozmístění jednotlivých prvků v konkrétní komponentě ani správnou adaptaci na velikost zařízení. Běžnou součástí aplikací jsou ikony. Ikony z Material Designu jsou ve Flutteru k přímému použití bez specifického nastavení. Programátor nemusí řešit žádné externí SVG soubory. Stejně to platí i pro použití fontů písem.

\subsection{SQLite}
Jedná se o velmi jednoduchou a rychlou relační databázi, která je uložena v jediném souboru \cite{sqlite}. Projekt SQLite byl zahájen v roce 2000. Obvykle jsou všechny potřebné závilosti již zabudovány v zařizeních, ať už jde o mobilní telefony nebo desktopové operační systémy. Je zdarma k užití pro libivolné účely. Přesto poskytuje plnohodnotnou SQL implementaci. Zajímavostí je, že onen soubor je multiplatformní. Nevadí mu přenos mezi 32 a 64 bitovými systémy nebo architekturami big-endian a little-endian.

\subsection{Balíčky}

Přestože balíčky tvoří závilosti projektu na ostatních okolnostech a programátor se musí spoléhat na jiné programátory, že je budou udržovat aktuální, je nemožné se jim v dnešním době vyhnout. Dart a Flutter má výběr z široké veřejné knihovny balíčků, které jsou vyvjíveny ostatními programátory. 

Balíčky je nutné do projektu zahrnout. Celý jejich seznam je dostupný v souboru \verb|pubspec.yaml|. 

\begin{itemize}
  \item \textbf{Drift} je balíček poskytující rozhraní pro práci s SQLite.  databází \cite{drift}. Její velkou výhodou oproti ostatním implementacím SQLite pro Dart je možnost multiplatformnosti. Funguje na všech platformách od Androidu přes Windows až po webové rozhraní. V ofiálním návodu je doporučena knihovna s názvem \textbf{sqflite} ta však tuto výhodu neposkytuje. Dále poskytuje velmi jednoduché API pro vytváření dotazů nad databází bez použití jazyka SQL. Programátor vytváří strukturu databáze v jednom souboru, na jehož základě si Drift generuje interní soubory.
  \item \textbf{GoRouter} je deklarativní balíček pro organiazci navigace v rámci aplikace (tzv. \textit{routování}). Funguje na principu URL adres. Umožňuje v adrese předávat parametry, zajistit přesměrování na základě práv uživatele či použití tzv. \textit{vnitřních navigátorů} nejčastěji pro komponentu \kiinlinecode{csharp}{;}{BottomNavigationBar}, který zůstává neustále viditelný skrz více obrazovek. Pro přechody umožňuje nastavit vlastní libovolné animace. Výhodou je, že autorem jsou přímo oficiální vývojaři Flutter, tudíž by balíček měl zůstat udržovaný.
  \item \textbf{Skeletonizer} zjednodušeným způsobem poskytuje funkcionalitu označovanou jako \textit{skeleton loading}. Používá se při načítáná obsahu na stránce a zobrazuje hrubý náhled rozložení prvků na stránce. Vše je doplněno vhodnou animací, aby uživatel ihned poznal, že se obrazovka právě načítá. Zjednodušení použitím tohoto balíčku spočívá v tom, že programátorovi stačí již definovaný widget \uv{obalit} widgetem \verb|Skeletonizer| z tohoto balíčku a o zbytek práce se balíček postará sám.
  \item \textbf{Another Flushbar} poskytuje vylepšenou komponentu \verb|Snackbar| z Material Design. Účelem komponenty je krátce informovat uživatele o akci, která byla právě provedena. Obvykle se objevuje ve spodní části obrazovky. Vylepšení spočítá v možnosti většího přizpůsobením - přidání ikony, změna barvy či animace a libovolné umístění. Já tento balíček preferoval i z důvodu správného zobrazování při otevřeném dialogovém okně.
  \item \textbf{Syncfusion Flutter Charts} je rozsáhlá knihovna pro tvorbu grafů. Umožňuje vytvářet všechny druhy grafů - sloupcové, spojnicové, prstencové a další (autoři uvádí více než 30 druhů). Grafy lze dále přizpůsobovat - výběr animace, úpráv jednotek na osách, výběr barev, možnost zobrazit legendu či podrobné informace daného bodu v grafu po přejetí myší. Přes širokou škálu úprav je základní použití velmi jednoduché. Velkým bonusem je rozsáhlá dokumentace s úkázkami všech typů grafů a jejich zdrojových kódů.
  \item \textbf{File Picker} dává programátorovi možnsot využít nativní průzkumník souborů k výběru složky nebo konkrétích souborů pro jejich další zpracování v aplikaci. Je možnost soubory filtrovat či umožnit výběr více z nich najednou. Základní funkcionalitu poskytuje na libovolné platformně s velmi jednoduchou implementací. V mém případě jsem ho použil pro výběr složky, kam si uživatel chce uložit exportované soubory. 

\end{itemize}


\section{Programátorská příručka}

Multiplatformní vývoj přináší jistá specifika (nutnost zvolit vhodné technologie, nutnost případné optimalizace zobrazení či zúžený výběr knihoven), ale z hlediska architektury či samothého procesu programování je většina věcí stejných. Zvolená technologie často sama navádí k použití vhodných prvků.

Pro vývoj aplikace jsem použil textový editor Visual Studio Code s řadou rozšíření. Část práce specifické pro platformu Android jsem realizoval v prostředí Android Studio. Především se jednalo o úlohy typu aktualizace Software Developement Kit či migrace na novější verzi Kotlin gradle, kde Androi Studio poskytuje jednodušší rozhraní a přesný návod.

\subsection{Architektura aplikace}

\subsection{Multiplatformní část projektu}


\subsection{Implementace responzivního designu}

\subsection{Neočekáváné zajímavosti při tvorbě práce}

Problém s Sqlite knihovnou.

Nedoděláné všechny komponenty material designu.


\section{Uživatelská příručka}

\subsection{První spuštění}

\subsection{Zobrazení a úprava transakcí}

\subsection{Využití reportů}

\subsection{Nastavení aplikace}

\subsection{Přizpůsobení vzhledu}

\subsection{Přizpůsobení měn a kategorií}

\subsection{Export dat}


\section{Možná rozšíření aplikace}
Osobně si myslím, že nikdy se nedá o softwaru prohlásit, že by nepotřeboval vývoj a není možné mu dodat vylepšení. Už jen vzhledem k vnějšímu vývoji technologií a proměně uživatelských požadavků je nutné aplikaci udržovat aktuální. Se změnou požadavků se pojí rozšíření o nové funkce a nebo naopak odstranění funkcí nepouživaných.

\begin{enumerate}
  \item Možnost založit více účtů
  \item Umožnit nastavení vlastní měny
  \item Import dat z jiných aplikací
  \item Vylepšení uživatelského rozhraní
\end{enumerate}

\begin{kiconclusions}
Tady bude závěr.
\end{kiconclusions}

\begin{kiconclusions}[english]
Here will be conclusion.
\end{kiconclusions}

% ----- tady začíná to co v práci bylo dáno

%%%  Po přeložení programem CSLaTeX (třikrát) je potřeba použít
%%%  program DVIPS a takto získaný PostScriptový soubor vytisknout
%%%  na PostScriptové tiskárně nebo pomocí programu GhostScript.
%%%
%%%  Rovněž je možné použít program DVIPDFM a vytvořit z dokumentu
%%%  soubor ve formátu PDF včetně hypertextových odkazů.

%%% Argument `joinlists' způsobí zřetězení seznamů obrázků, tabulek,
%%% vět a zdrojových kódů. Není-li použít, všechny seznamy jsou
%%% uvedeny na samostatných stránkách.

%%  'encoding=kódování' pro kódování tohoto a vložených zdrojových
%%  textů v kódování jiném než výchozím utf8

%% -------------------------------------------------------------------

%% Sazba povinné bibliografie, za přílohami (případně i za seznamem
%% zkratek). Při použití BibLaTeXu použijte makro
%% \printbibliography. jinak prostředí thebibliography. Ne obojí!

%% Sazba i v textu necitovaných zdrojů, při použití
%% BibLaTeXu. Volitelné.
%% \nocite{*}
%% Vlastní sazba bibliografie při použití BibLaTeXu.
\printbibliography
\nocite{*}


%% Sazba volitelného rejstříku, za bibliografií.
%% \printindex
%% \printglossaries
\end{document}
