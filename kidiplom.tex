%%%  Šablona práce od:
%%%  Copyright (C) 2012 Martin Rotter, <rotter.martinos@gmail.com>
%%%  Copyright (C) 2014 Jan Outrata, <jan.outrata@upol.cz>

%%  Pro získání PDF souboru dokumentu je třeba tento zdrojový text v
%%  LaTeXu přeložit (dvakrát) programem pdfLaTeX.

%%  V případě použití programu BibLaTeX pro tvorbu seznamu literatury
%%  je poté ještě třeba spustit program Biber s parametrem jméno
%%  souboru zdrojového textu bez přípony a následně opět (dvakrát)
%%  přeložit zdrojový text programem pdfLaTeX.

%%  Postup získání Postscriptového souboru je popsán v dokumentaci.

%%  Třída dokumentu implementující styl pro závěrečnou práci. Vybrané
%%  nepovinné parametry (ostatní v dokumentaci):

%%  'master' pro sazbu diplomové práce, jinak se sází bakalářská práce

%%  'program=kód' pro Váš studijní program/obor (specializaci), kódy
%%  pro diplomovou práci 'infoi' pro Informatiku (Obecná informatika),
%%  'infui' pro Informatiku (Umělá inteligence), 'ainfpst' pro
%%  Aplikovanou informatiku (Počítačové systémy a technologie), 'uinf'
%%  pro Učitelství informatiky pro střední školy, 'binf' pro
%%  Bioinformatiku, 'inf' pro Informatiku (bez specializací) a 'ainf'
%%  pro Aplikovanou informatiku (bez specializací), jinak je výchozí
%%  ainfvs pro Aplikovanou informatiku (Vývoj software), a pro
%%  bakalářskou práci 'infoi' pro Informatiku (Obecná informatika),
%%  'itp' pro Informační technologie v prezenční formě, 'itk' pro
%%  Informační technologie v kombinované formě, 'infv' pro Informatiku
%%  pro vzdělávání, 'binf' pro Bioinfomatiku, 'inf' pro Informatiku
%%  (bez specializací), 'ainfp' pro Aplikovanou informatiku (bez
%%  specializací) v prezenční formě, 'ainfk' pro Aplikovanou
%%  informatiku (bez specializací) v kombinované formě, jinak je
%%  výchozí infpvs pro Informatiku (Programování a vývoj software)

%%  'printversion' pro sazbu verze pro tisk (nebarevné logo a odkazy,
%%  odkazy s uvedením adresy za odkazem, ne odkazy do rejstříku),
%%  jinak verze pro prohlížeč

%%  'biblatex' pro zapnutí podpory pro sazbu bibliografie pomocí
%%  BibLaTeXu, jinak je výchozí sazba v prostředí thebibliography

%%  'language=jazyk' pro jazyk práce, jazyky english pro anglický,
%%  slovak pro slovenský, jinak je výchozí czech pro český

%%  'font=sans' pro bezpatkový font (Iwona Light), jinak je výchozí
%%  serif pro patkový (Latin Modern)

%%  'figures, tables, theorems a sourcecodes' pro sazbu seznamu
%%  obrázků, tabulek, vět a zdrojových kódů, jinak při =false se
%%  nesází (u theorems a sourcecodes výchozí)

\documentclass[
%  master,
%  program=ainfvs,
%  printversion,
  biblatex,
%  language=english,
%  font=sans,
  figures=false,
  tables=false,
%  theorems,
%  sourcecodes,
  glossaries,
  index
]{kidiplom}

%% Informace pro úvodní strany. V jazyku práce (pokud není v komentáři
%% uvedeno česky) a anglicky. Uveďte všechny, u kterých není v
%% komentáři uvedeno, že jsou volitelné. Při neuvedení se použijí
%% výchozí texty. Text pro jiný než nastavený jazyk práce (nepovinným
%% parametrem language makra \documentclass, výchozí český) se zadává
%% použitím makra s uvedením jazyka jako nepovinného parametru.

%% Název práce, česky a anglicky. Měl by se vysázet na jeden řádek.
\title{Multiplatformní aplikace pro správu osobních financí}
\title[english]{Cross-platform application for personal finance management}

%% Jméno autora práce. Makro nemá nepovinný parametr pro uvedení
%% jazyka.
\author{Vojtěch Netrh}

%% Jméno vedoucího práce (včetně titulů). Makro nemá nepovinný
%% parametr pro uvedení jazyka.
\supervisor{doc. RNDr. Jan Konečný, Ph.D.}

%% Volitelný rok odevzdání práce. Výchozí je aktuální (kalendářní)
%% rok. Makro nemá nepovinný parametr pro uvedení jazyka.
\yearofsubmit{2025}

%% Anotace práce, včetně anglické (obvykle překlad z jazyka
%% práce). Jeden odstavec!
\annotation{Ukázkový text závěrečné práce na Katedře informatiky
  Přírodovědecké fakulty Univerzity Palackého v Olomouci, který je
  zároveň dokumentací stylu pro text práce v \LaTeX{}u. Zdrojový text
  v \LaTeX{}u je doporučeno použít jako šablonu pro text skutečné
  závěrečné práce studenta.}

\annotation[english]{Sample text of thesis at the \kitextdepten,
  \kitextfacultyen, \kitextuniven{} and, at the same time,
  documentation of the \LaTeX{} style for the text. The source text in
  \LaTeX{} is recommended to be used as a template for real student's
  thesis text.}

%% Klíčová slova práce, včetně anglických. Oddělená středníkem.
\keywords{Flutter; Dart; multiplatformní; osobní finance}
\keywords[english]{Flutter; Dart; cross-platform; personal finance}

%% Volitelná specifikace příloh textu práce, i anglicky. Výchozí je
%% 'elektronická data v systému katedry informatiky / electronic data
%% in system of department of computer science'.
%\supplements{nejlepší software všech dob}
%\supplements[english]{the best software of all times}

%% Volitelné poděkování. Stručné! Výchozí je prázdné. Makro nemá
%% nepovinný parametr pro uvedení jazyka.
\thanks{Děkuji panu doc. RNDr. Janu Konečnému Ph. D. za cenné rady a podněty při tvorbě práce. Svým blízkým za podporu běheme celého bakalářského studia.}

%% Cesta k souboru s bibliografií pro její sazbu pomocí BibLaTeXu
%% (zvolenou nepovinným parametrem biblatex makra
%% \documentclass). Použijte pouze při této sazbě, ne při (výchozí)
%% sazbě v prostředí thebibliography.
\bibliography{bibliografie.bib}

%% Další dodatečné styly (balíky) potřebné pro sazbu vlastního textu
%% práce.
\usepackage{lipsum}
\usepackage{longtable}

\begin{document}
%% Sazba úvodních stran -- titulní, s bibliografickými údaji, s
%% anotací a klíčovými slovy, s poděkováním a prohlášením, s obsahem a
%% se seznamy obrázků, tabulek, vět a zdrojových kódů (pokud jejich
%% sazba není vypnutá).
\maketitle

%% Vlastní text závěrečné práce. Pro povinné závěry, před přílohami,
%% použijte prostředí kiconclusions. Povinná je i příloha s obsahem
%% elektronických dat.

%% -------------------------------------------------------------------

\newcommand{\BibLaTeX}{\textsc{Bib}\LaTeX}

% ----- tady začíná MNOU PSANÝ TEXT

\section{Úvod}
Peníze se vyskytují všude kolem nás a chceme-li nebo ne, hrají v našich životech podstatnou roli. Z hlediska osobního pohledu jednoho člověka je užitečné mít ve svých financích pořádek. Můžeme tak ušetřit peníze, případně je efektivněji využívat.

\subsection{Motivace}
Požadované funkcionality na tuto aplikaci se současně odvíjely ode mě i mého vedoucího pane Konečného. Oba jsme si představovali mít jednoduchou aplikaci pro správu našich finančních prostředků.

\subsection{Požadavky}

\section{Přehled existujících řešení}
K danému tématu již exstuje řada aplikací nabízejících nástroje pro správu financí. Každé řešení přistupuje k problému jiným způsobem. 

Nejblíže je z hlediska uživatele poskytnutí základních nástrojů přímo v bankovní aplikaci. Z mého pohledu je to nejméně vhodné řešení hned z několika důvodů:
\begin{itemize}
  \item obvykle má člověk účet u více bank (aplikace jedné z nich neposkytuje přehled o celkových financích),
  \item jen část (ikdyž v dnešní době většinová) transakcí je prováděna pomocí platební karty,
  \item uživatel nemusí chtít mít všechny pohyby na účtu zahrnuty do celkové analýzy.
\end{itemize}

Dalším odvětví jsou aplikace třetích stran, které nabízí různé možnosti na základě daného typu a zaměření.

\subsection{Wallet}
Mobilní aplikace dostupné na platformy Android i iOS, za jejíž stažení uživatel v první fázi nezaplatí.

\subsection{1Money}
Ještě na konci roku 2024 obilní aplikace dostupná na obě platformy, o pár měsícu později už pouze na iOS. Z mnou vyzkoušených aplikací největší favorit pro denní užití. Aplikace mě zaujala svojí obrazovkou \textit{Categories}. Název nepůsobí nijak zajimavě, ale zpracování ano. Středem obrazovky je prstencový graf zobrazující poměr utracených peněz podle kategorií a v jeho středu se ještě nachází dva údaje - stav příjmů a výdajů. Zbytek obrazovky pokrývají kolečka označující kategorie, po jejichž stisknutí je uživateli ihned umožněno přidat transakci s danou kategorií a aktuálním časovým razítkem. Velmi intuitivní a rychlé řešení pro každodenní použití.

\subsection{GnuCash}

\subsection{Homebank}


\section{Použité technologie}

Požadavkem byla multiplatformní vývoj existuje řada technologií, které však k tvorbě aplikace přistupují odlišným způsobem. Část technologií jde cestou dnes velmi moderních webových aplikací, které se tváří jako běžné aplikace. Primárně tedy používají JavaScript jakožto programovací jazyk jak pro část logickou, tak i pro uživatelská rozhraní. Druhý přístup, který jsem zvolil já, jde více cestou desktopových aplikací. 

Aplikace je otestována a řádně funguje na obou mobilních platformách (Android iOS) a jako desktopová Windows aplikace.

\subsection{Dart}

\subsection{Flutter}
Tohle je test citace \cite{flutter}

\subsection{Material Design}

\subsection{SQLite}

\subsection{Knihovny}


\section{Programátorská příručka}

Multiplatformní vývoj přináší jistá specifika (nutnost zvolit vhodné technologie, nutnost případné optimalizace zobrazení či zúžený výběr knihoven), ale z hlediska architektury či samothého procesu programování je většina věcí stejných. Zvolená technologie často sama navádí k použití vhodných prvků.

Pro vývoj aplikace jsem použil textový editor Visual Studio Code s řadou rozšíření. Část práce specifické pro platformu Android jsem realizoval v prostředí Android Studio. Především se jednalo o úlohy typu aktualizace Software Developement Kit či migrace na novější verzi Kotlin gradle, kde Androi Studio poskytuje jednodušší rozhraní a přesný návod.

\subsection{Architektura aplikace}

\subsection{Multiplatformní část projektu}


\subsection{Implementace responzivního designu}

\subsection{Neočekáváné zajímavosti při tvorbě práce}

Problém s Sqlite knihovnou.

Nedoděláné všechny komponenty material designu.


\section{Uživatelská příručka}

\subsection{První spuštění}

\subsection{Zobrazení a úprava transakcí}

\subsection{Využití reportů}

\subsection{Export dat}

\subsection{Nastavení aplikace}

\subsection{Přizpůsobení vzhledu}


\section{Možná rozšíření aplikace}
Osobně si myslím, že nikdy se nedá o softwaru prohlásit, že by nepotřeboval vývoj a není možné mu dodat vylepšení. Už jen vzhledem k vnějšímu vývoji technologií a proměně uživatelských požadavků je nutné aplikaci udržovat aktuální. Se změnou požadavků se pojí rozšíření o nové funkce a nebo naopak odstranění funkcí nepouživaných.

\begin{enumerate}
  \item Možnost založit více účtů
  \item Umožnit nastavení vlastní měny
  \item Import dat z jiných aplikací
  \item Vylepšení uživatelského rozhraní
\end{enumerate}

\begin{kiconclusions}
Tady bude závěr.
\end{kiconclusions}

\begin{kiconclusions}[english]
Here will be conclusion.
\end{kiconclusions}

% ----- tady začíná to co v práci bylo dáno

%%%  Po přeložení programem CSLaTeX (třikrát) je potřeba použít
%%%  program DVIPS a takto získaný PostScriptový soubor vytisknout
%%%  na PostScriptové tiskárně nebo pomocí programu GhostScript.
%%%
%%%  Rovněž je možné použít program DVIPDFM a vytvořit z dokumentu
%%%  soubor ve formátu PDF včetně hypertextových odkazů.

%%% Argument `joinlists' způsobí zřetězení seznamů obrázků, tabulek,
%%% vět a zdrojových kódů. Není-li použít, všechny seznamy jsou
%%% uvedeny na samostatných stránkách.

%%  'encoding=kódování' pro kódování tohoto a vložených zdrojových
%%  textů v kódování jiném než výchozím utf8

%% -------------------------------------------------------------------

%% Sazba povinné bibliografie, za přílohami (případně i za seznamem
%% zkratek). Při použití BibLaTeXu použijte makro
%% \printbibliography. jinak prostředí thebibliography. Ne obojí!

%% Sazba i v textu necitovaných zdrojů, při použití
%% BibLaTeXu. Volitelné.
%% \nocite{*}
%% Vlastní sazba bibliografie při použití BibLaTeXu.
\printbibliography
\nocite{*}


%% Sazba volitelného rejstříku, za bibliografií.
%% \printindex
\printglossaries % Add glossary printing command
\end{document}
