%%%  Ukázkový text a dokumentace stylu pro text závěrečné (bakalářské a
%%%  diplomové) práce na KI PřF UP v Olomouci
%%%  Copyright (C) 2012 Martin Rotter, <rotter.martinos@gmail.com>
%%%  Copyright (C) 2014 Jan Outrata, <jan.outrata@upol.cz>


%%  Pro získání PDF souboru dokumentu je třeba tento zdrojový text v
%%  LaTeXu přeložit (dvakrát) programem pdfLaTeX.

%%  V případě použití programu BibLaTeX pro tvorbu seznamu literatury
%%  je poté ještě třeba spustit program Biber s parametrem jméno
%%  souboru zdrojového textu bez přípony a následně opět (dvakrát)
%%  přeložit zdrojový text programem pdfLaTeX.

%%  Postup získání Postscriptového souboru je popsán v dokumentaci.


%%  Třída dokumentu implementující styl pro závěrečnou práci. Vybrané
%%  nepovinné parametry (ostatní v dokumentaci):

%%  'master' pro sazbu diplomové práce, jinak se sází bakalářská práce

%%  'program=kód' pro Váš studijní program/obor (specializaci), kódy
%%  pro diplomovou práci 'infoi' pro Informatiku (Obecná informatika),
%%  'infui' pro Informatiku (Umělá inteligence), 'ainfpst' pro
%%  Aplikovanou informatiku (Počítačové systémy a technologie), 'uinf'
%%  pro Učitelství informatiky pro střední školy, 'binf' pro
%%  Bioinformatiku, 'inf' pro Informatiku (bez specializací) a 'ainf'
%%  pro Aplikovanou informatiku (bez specializací), jinak je výchozí
%%  ainfvs pro Aplikovanou informatiku (Vývoj software), a pro
%%  bakalářskou práci 'infoi' pro Informatiku (Obecná informatika),
%%  'itp' pro Informační technologie v prezenční formě, 'itk' pro
%%  Informační technologie v kombinované formě, 'infv' pro Informatiku
%%  pro vzdělávání, 'binf' pro Bioinfomatiku, 'inf' pro Informatiku
%%  (bez specializací), 'ainfp' pro Aplikovanou informatiku (bez
%%  specializací) v prezenční formě, 'ainfk' pro Aplikovanou
%%  informatiku (bez specializací) v kombinované formě, jinak je
%%  výchozí infpvs pro Informatiku (Programování a vývoj software)

%%  'printversion' pro sazbu verze pro tisk (nebarevné logo a odkazy,
%%  odkazy s uvedením adresy za odkazem, ne odkazy do rejstříku),
%%  jinak verze pro prohlížeč

%%  'biblatex' pro zapnutí podpory pro sazbu bibliografie pomocí
%%  BibLaTeXu, jinak je výchozí sazba v prostředí thebibliography

%%  'language=jazyk' pro jazyk práce, jazyky english pro anglický,
%%  slovak pro slovenský, jinak je výchozí czech pro český

%%  'font=sans' pro bezpatkový font (Iwona Light), jinak je výchozí
%%  serif pro patkový (Latin Modern)

%%  'figures, tables, theorems a sourcecodes' pro sazbu seznamu
%%  obrázků, tabulek, vět a zdrojových kódů, jinak při =false se
%%  nesází (u theorems a sourcecodes výchozí)

\documentclass[
%  master,
%  program=ainfvs,
%  printversion,
  biblatex,
%  language=english,
%  font=sans,
  figures=false,
%  tables=false,
%  theorems,
%  sourcecodes,
  glossaries,
  index
]{kidiplom}

%% Informace pro úvodní strany. V jazyku práce (pokud není v komentáři
%% uvedeno česky) a anglicky. Uveďte všechny, u kterých není v
%% komentáři uvedeno, že jsou volitelné. Při neuvedení se použijí
%% výchozí texty. Text pro jiný než nastavený jazyk práce (nepovinným
%% parametrem language makra \documentclass, výchozí český) se zadává
%% použitím makra s uvedením jazyka jako nepovinného parametru.

%% Název práce, česky a anglicky. Měl by se vysázet na jeden řádek.
\title{Multiplatformní aplikace pro správu osobních financí}
\title[english]{Cross-platform application for personal finance management}

%% Volitelný podnázev práce, česky a anglicky. Měl by se vysázet na
%% jeden řádek. Výchozí je prázdný.
%% \subtitle{Ukázkový text a dokumentace stylu v \LaTeX{}u}
%% \subtitle[english]{Sample text and documentation of the \LaTeX{} style}

%% Jméno autora práce. Makro nemá nepovinný parametr pro uvedení
%% jazyka.
\author{Vojtěch Netrh}

%% Jméno vedoucího práce (včetně titulů). Makro nemá nepovinný
%% parametr pro uvedení jazyka.
\supervisor{doc. RNDr. Jan Konečný, Ph.D.}

%% Volitelný rok odevzdání práce. Výchozí je aktuální (kalendářní)
%% rok. Makro nemá nepovinný parametr pro uvedení jazyka.
\yearofsubmit{2025}

%% Anotace práce, včetně anglické (obvykle překlad z jazyka
%% práce). Jeden odstavec!
\annotation{Ukázkový text závěrečné práce na Katedře informatiky
  Přírodovědecké fakulty Univerzity Palackého v Olomouci, který je
  zároveň dokumentací stylu pro text práce v \LaTeX{}u. Zdrojový text
  v \LaTeX{}u je doporučeno použít jako šablonu pro text skutečné
  závěrečné práce studenta.}

\annotation[english]{Sample text of thesis at the \kitextdepten,
  \kitextfacultyen, \kitextuniven{} and, at the same time,
  documentation of the \LaTeX{} style for the text. The source text in
  \LaTeX{} is recommended to be used as a template for real student's
  thesis text.}

%% Klíčová slova práce, včetně anglických. Oddělená (obvykle) středníkem.
\keywords{Flutter; Dart; multiplatformní; osobní finance}
\keywords[english]{Flutter; Dart; cross-platform; personal finance}

%% Volitelná specifikace příloh textu práce, i anglicky. Výchozí je
%% 'elektronická data v systému katedry informatiky / electronic data
%% in system of department of computer science'.
%\supplements{nejlepší software všech dob}
%\supplements[english]{the best software of all times}

%% Volitelné poděkování. Stručné! Výchozí je prázdné. Makro nemá
%% nepovinný parametr pro uvedení jazyka.
\thanks{Děkuji panu doc. RNDr. Janu Konečnému Ph. D. za cenné rady a podporu v průběhu tvorby práce. V neposlední řadě děkuji rodině a přítelkyni za trpělivost v průběhu celé tvorby práce.}

%% Cesta k souboru s bibliografií pro její sazbu pomocí BibLaTeXu
%% (zvolenou nepovinným parametrem biblatex makra
%% \documentclass). Použijte pouze při této sazbě, ne při (výchozí)
%% sazbě v prostředí thebibliography.
\bibliography{bibliografie.bib}

%% Další dodatečné styly (balíky) potřebné pro sazbu vlastního textu
%% práce.
\usepackage{lipsum}
\usepackage{longtable}

\begin{document}
%% Sazba úvodních stran -- titulní, s bibliografickými údaji, s
%% anotací a klíčovými slovy, s poděkováním a prohlášením, s obsahem a
%% se seznamy obrázků, tabulek, vět a zdrojových kódů (pokud jejich
%% sazba není vypnutá).
\maketitle

%% Vlastní text závěrečné práce. Pro povinné závěry, před přílohami,
%% použijte prostředí kiconclusions. Povinná je i příloha s obsahem
%% elektronických dat.

%% -------------------------------------------------------------------

\newcommand{\BibLaTeX}{\textsc{Bib}\LaTeX}

% \noindent\textcolor{red}{\LARGE Upozornění: Následující text
%   dokumentace stylu, vyjma přílohy~\ref{sec:ObsahData}, je rozpracovaná
%   a (značně) neúplná verze!!!}

% ----- tady začíná  MŮJ TEXT

\section{Úvod}
Peníze se vyskytují všude kolem nás a chceme-li nebo ne, hrají v našich životech podstatnou roli. Z hlediska osobního pohledu jednoho člověka je užitečné mít ve svých financích pořádek. Můžeme tak ušetřit peníze, případně je efektivněji využívat.

\subsection{Požadavky na tvorbu}
Požadované funkcionality na tuto aplikaci se současně odvíjely ode mě i mého vedoucího pane Konečného. Oba jsme si představovali mít jednoduchou aplikaci pro správu našich finančních prostředků.  




% ----- tady začíná to co v práci bylo dáno

\section{Styly pro psaní bakalářských a diplomových prací}
Toto jsou styly pro psaní bakalářských a diplomových prací přes typografický systém \LaTeX{}, tedy \textbf{kistyles}.

\subsection{Požadavky a podprovaná prostředí}
Sada balíku \textbf{kistyles} podporuje následující distribuce systému \LaTeX{}:
\begin{itemize}
\item \TeX{} Live.
\end{itemize}

Jsou podporovány všechny výstupní ovladače, tedy jak \textbf{dvi}, tak \textbf{pdf} i \textbf{ps}. Funkčnost zmiňovaných distribucí byla ověřena na několika operačních systémech, mezi které patří:
\begin{enumerate}
\item Windows $8.1$,
\item Archlinux,
\item Debian GNU/Linux.
\end{enumerate}

Důrazně se doporučuje používat aktuální verzi dané distribuce systému \LaTeX{}.

%%%  Po přeložení programem CSLaTeX (třikrát) je potřeba použít
%%%  program DVIPS a takto získaný PostScriptový soubor vytisknout
%%%  na PostScriptové tiskárně nebo pomocí programu GhostScript.
%%%
%%%  Rovněž je možné použít program DVIPDFM a vytvořit z dokumentu
%%%  soubor ve formátu PDF včetně hypertextových odkazů.

\subsection{Přepínače}
Styl kidiplom je z hlediska uživatele zastoupen ekvivalentně nazvanou třídou, kterou je třeba volat na záčátku dokumentu:
\begin{kicode}{TeX}{}{Volání třídy \textbf{kidiplom}}
\documentclass[
  master,
  program=ainfvs,
  printversion,
  biblatex,
  language=english,
  font=sans,
  figures=false,
  tables=false,
  theorems,
  sourcecodes,
  joinlists,
  glossaries,
  index,
  encoding=utf8,
  bibencoding=utf8
]{kidiplom}
\end{kicode}

Následuje přehled přepínačů, je vždy uvedeno jméno přepínač, včetně výchozí hodnoty. Přepínače uvádí tabulka \ref{tab:prepinace}.

%\begin{table}
\begin{center}
\begin{longtable}{>{\bfseries}l >{\ttfamily}c L{8cm}}
\caption{Seznam přepínačů}\label{tab:prepinace}\\
  {\normalfont Přepínač} & {\normalfont Výchozí hodnota} & {\normalfont Popis} \\
\hline
master & false & Povolí nebo zakáže režim diplomové práce. Výchozí režim je tedy bakalářská práce. \\

printversion & false & Je-li zapnuto, pak budou odkazy vysázeny optimalizovaně pro knižní sazbu. Tuto volbu je nutno použít pro tisk práce. \\

biblatex & false & Zapne sazbu bibliografie přes balík \BibLaTeX{}. \\

language & czech & Jazyk textu práce. Možné hodnoty jsou \textbf{czech},
\textbf{english} a \textbf{slovak}. \\

font & serif & Zapne či vypne podporu pěkného bezpatkového fontu. Možné hodnoty jsou:\newline
\begin{description}\setlength{\itemsep}{-1ex}
\item[serif] Patkové písmo (Computer Modern).
\item[sans] Bezpatkové písmo (Iwona).
\end{description} \\[-3ex]

figures & true & Je-li zapnuto, pak v seznamech položek bude zahrnut seznam obrázků. \\

tables & true & Je-li zapnuto, pak v seznamech položek bude zahrnut seznam tabulek. \\

theorems & false & Je-li zapnuto, pak v seznamech bude zahrnut seznam teorémů. \\

sourcecodes & false & Je-li zapnuto, pak v seznamech bude zahrnut seznam zdrojových kódů. \\

%%% Argument `joinlists' způsobí zřetězení seznamů obrázků, tabulek,
%%% vět a zdrojových kódů. Není-li použít, všechny seznamy jsou
%%% uvedeny na samostatných stránkách.

joinlists & true & Je-li zapnuto, pak seznamy obrázků, tabulek, vět a
zdrojových kódů sázené za obsahem nebudou rozděleny na samostatné stránky. \\

glossaries & false & Je-li zapnuto, pak na konci dokumentu bude vysázen seznam zkratek. \\

index & false & Zapíná podporu sazby rejstříku. \\

%%  'encoding=kódování' pro kódování tohoto a vložených zdrojových
%%  textů v kódování jiném než výchozím utf8
encoding & utf8 & Kódování souboru dokumentu, doporučuje se ponechat výchozí hodnotu. \\

bibencoding & utf8 & Kódování souboru bibliografie. Tato volba má
smysl pouze, pokud je použita bibliografie skrze balíček \BibLaTeX{}. \\

program & \vtop{\hbox{\strut infpvs}\hbox{\strut ainfvs}} & Specifikuje studijní program/obor (specializaci):\newline
\begin{description}
\item[infoi] Informatika (Obecná informatika)\,--\,bakalářský i navazující magisterský,
\item[infpvs] Informatika (Programování a vývoj software)\,--\,bakalářský,
\item[itp] Informační technologie\,--\,bakalářský, prezenční forma,
\item[itk] Informační technologie\,--\,bakalářský, kombinovaná forma,
\item[infui] Informatika (Umělá inteligence)\,--\,navazující magisterský,
\item[ainfvs] Aplikovaná informatika (Vývoj software)\,--\,navazující magisterský,
\item[ainfpst] Aplikovaná informatika (Počítačové systémy a technologie)\,--\,navazující magisterský,
\item[infv] Informatika pro vzdělávání\,--\,bakalářský,
\item[uinf] Učitelství informatiky pro střední školy\,--\,navazující magisterský,
\item[binf] Bioinformatika\,--\,bakalářský i navazující magisterský,
\item[inf] Informatika (bez specializací)\,--\,bakalářský i navazující magisterský,
\item[ainfp] Aplikovaná informatika (bez specializací)\,--\,bakalářský, prezenční forma,
\item[ainfk] Aplikovaná informatika (bez specializací)\,--\,bakalářský, kombinovaná forma,
\item[ainf] Aplikovaná informatika (bez specializací)\,--\,navazující magisterský.
\end{description}
\end{longtable}
\end{center}
%\end{table}

\subsection{Geometrie stránky}
Tento styl používá list velikosti $A4$. Pro sazbu prací je třeba použít jednostrannou sazbu. Levý okraj je rozšířen s ohledem na vazbu výsledné knižní podoby práce.

\section{Sazba částí dokumentu}
\subsection{Sazba úvodní strany či obsahu}
Vysázení všech podstatných částí úvodu práce obstará makro \kiinlinecode{TeX}{!}{\\maketitle}. Pro správné vysázení všech částí a meta-informací je potřeba použí makra \kiinlinecode{TeX}{!}{\\title}, \mbox{\kiinlinecode{TeX}{!}{\\author}} a další. Jejich přehled lze najít ve zdrojovém souboru tohoto dokumentu. V případě použítí \textbf{pdf} výstupu se generuje i dodatečná hlavička souboru s meta-informacemi jako je autor dokumentu, název práce či dalšími.

\subsection{Závěry}
Závěr práce by se měl poskytnout jak v původním jazyce práce, tak v jazyce anglickém. Pro sazbu závěru jsou k dispozici příslušná makra. Berte na vědomí, že v anglickém závěru se aktivuje plně anglická sazba se všemi konvencemi. Tedy je třeba používat anglické uvozovky a další správné typografické prvky.

\begin{kicode}{TeX}{}{Sazba závěrů}
% Tiskne český závěr práce.
\begin{kiconclusions}
Závěr práce v \uv{českém} jazyce.
\end{kiconclusions}

% Tiskne anglický závěr práce.
\begin{kiconclusions}[english]
Thesis conclusions written in \uv{English}.
\end{kiconclusions}
\end{kicode}

\subsection{Matematika}
Pro sazbu matematiky je k dispozici sada standardních maker.
$$\langle f \rangle, \lfloor g \rfloor,
\lceil h \rceil, \ulcorner i \urcorner$$

$$\left\{\frac{x^2}{y^3}\right\}$$

$$
A_{m,n} =
 \begin{pmatrix}
  a_{1,1} & a_{1,2} & \cdots & a_{1,n} \\
  a_{2,1} & a_{2,2} & \cdots & a_{2,n} \\
  \vdots  & \vdots  & \ddots & \vdots  \\
  a_{m,1} & a_{m,2} & \cdots & a_{m,n}
 \end{pmatrix}
$$

$$
M = \begin{bmatrix}
       \frac{5}{6} & \frac{1}{6} & 0           \\[0.3em]
       \frac{5}{6} & 0           & \frac{1}{6} \\[0.3em]
       0           & \frac{5}{6} & \frac{1}{6}
     \end{bmatrix}
$$

\subsection{Sazba literatury}
Pro sazbu literatury má uživatel dvě možnosti. Může použít služeb balíků \BibLaTeX{}, který je pro \textbf{kistyles} zapnutý, či lze použít manuální sazbu bibliografie.
\subsubsection{Sazba bibliografie přes \BibLaTeX{}}
Při použití tohoto balíku se data o použité literatuře ukládají do dedikovaného textového souboru, ukázku najdete i v tomto stylu pod jménem \kiinlinecode{text}{!}{bibliografie.bib}.

Formát daného souboru je nad rámec této dokumentace a je na každém uživateli, aby si jej nastudoval. Bibliografie se tiskne makrem \kiinlinecode{TeX}{!}{\\printbibliography}. Taktéž v preambuli dokumentu je třeba definovat, který soubor data bibliografie obsahuje, tedy například \kiinlinecode{TeX}{!}{\\bibliography\{bibliografie.bib\}}.

Dokument, který využívá \BibLaTeX{} je následně nutné přeložit jak pomocí překladače zvoleného ovladače, tak pomocí aplikace \kiinlinecode{text}{!}{biber}. Více informací poskytne soubor \kiinlinecode{text}{!}{Makefile} z distribuce tohoto stylu.

Výhodou tohoto přístupu je, že bibliografie se vysází automaticky a (obvykle) není třeba manuální úprava formátování.

\subsubsection{Manuální sazba bibliografie}
Manuální sazba obnáší vysázení prostředí \kiinlinecode{text}{!}{thebibliography} ručně. To je nad rámec tohoto dokumentu. Ukázku tohoto přístupu lze samozřejmě nalézt ve zdrojovém souboru tohoto dokumentu nebo také \href{http://www.math.uiuc.edu/~hildebr/tex/bibliographies.html}{zde}.

Pro aktivaci manuální sazby bibliografie je třeba volat třídu \kiinlinecode{text}{!}{kidiplom} s parametrem \kiinlinecode{text}{!}{biblatex=false}. Mějte, prosím, na paměti, že v tomto módu jsou makra \kiinlinecode{text}{!}{\\bibliography} a \kiinlinecode{text}{!}{\\printbibliography} nedostupná.

\subsection{Drobná makra}
Základní styl definuje hned několik maker pro usnadnění práce. Například makro \kiinlinecode{TeX}{!}{\\buno} vysází řetezec \uv{bez újmy na obecnosti}. Je k dispozici i verze s prvním velkým písmenem, \kiinlinecode{TeX}{!}{\\Buno}.

Je rovněž možno přidávat položky do seznamu zkratek. K tomu slouží makro \mbox{\kiinlinecode{TeX}{!}{\\newacronym}}, které lze použít například jednoduše jako \kiinlinecode{TeX}{!}{\\newacronym\{UPOL\}\{UPOL\}\{\\kitextunivcz\}}. Na danou zkratku se pak lze odkazovat jednoduše, \mbox{\kiinlinecode{TeX}{!}{\\gls\{UPOL\}}}.

Sazba uvozovek respektuje nastavení částí dokumentu, a proto se doporučuje používat makro \kiinlinecode{TeX}{!}{\\uv}. V anglické závěru práce toto platí taky, viz tato PDF ukázka.

Styl podporuje sazbu odstavců v tabulkách, více obsahuje tabulka \ref{tab:odstavce}.

\begin{table}
\begin{center}
\caption{Odstavce v tabulkách}\label{tab:odstavce}
\begin{tabular}{L{4cm}|R{4cm}|L{4cm}}
\lipsum[23] & \lipsum[22] & \lipsum[21]
\end{tabular}
\end{center}
\end{table}

K dispozici jsou také makra pro sazbu \csharp{} (\kiinlinecode{TeX}{!}{\\csharp}) či \cpp{} (\kiinlinecode{TeX}{!}{\\cpp}).

%% v případě tvorby rejstříku přeložit vygenerovaný soubor .idx
%% programem Makeindex a v případě tvorby seznamu zkratek spustit
%% program Makeglossaries s parametrem jméno souboru zdrojového textu
%% bez přípony a následně opět (dvakrát) přeložit zdrojový text
%% programem pdfLaTeX.

\subsection{Sazba rejstříku}
Sazba rejstříku sestává z několika kroků:

\begin{enumerate}
\item Je třeba přes volbu \kiinlinecode{TeX}{!}{index=true} rejstříkování povolit.
\item Použítím makra \kiinlinecode{TeX}{!}{\\index} rejstříkovat vybrané pojmy.
\item Kompilovat s použitím utility \kiinlinecode{TeX}{!}{makeindex}. Pro specifika tohoto kroku si stačí prohlédnout soubor \kiinlinecode{text}{!}{Makefile}.
\end{enumerate}

Makro \kiinlinecode{TeX}{!}{\\index} je redefinováno tak, že sází klikací odkaz na výraz v rejstříku. Je doporučeno jej použít ihned za výrazem\index{výraz}.

\textbf{Omezení redefinovaného makra \kiinlinecode{TeX}{!}{\\index}}: klikací odkaz nefunguje, pokud použijete konstrukci \kiinlinecode{TeX}{!}{\\index\{výraz|makro\}} (resp. \kiinlinecode{TeX}{!}{\\index\{výraz|(makro\}}), např. \kiinlinecode{TeX}{!}{\\index\{výraz|textit\}}.

Rejstřík lze vysázet pomocí makra \kiinlinecode{TeX}{!}{\\printindex}.

\subsection{Sazba zdrojových kódů}
Styl nabízí dva způsoby sazby zdrojových kódů:

\begin{enumerate}
\item Sazbu řádkových kódů, například \kiinlinecode{CSS}{!}{background-color: white;}. K tomu slouží makro formátu \kiinlinecode{TeX}{!}{\\kiinlinecode\{jazyk\}\{separátor\}\{kód\}}. Za separátor je vhodné volit jakýkoliv znak, který se nevyskytuje v samotném sázeném zdrojovém kódu. Za jazyk je nutno dosadit jeden z těchto: C, TeX, PHP, HTML, Lisp, SQL, TeX, Python, Java, TutorialD, text, csharp, cpp, JavaScript, CSS.

\item Sazbu zdrojových kódu do separátních prostředí. Takto vytištěný kód se objeví v seznamu zdrojových kódů. Ukázka například zdrojový kód \ref{kod:cpp}. Ukázku sazby naleznete ve zdrojovém kódu tohoto dokumentu.
\end{enumerate}

\newacronym{UPOL}{UPOL}{\kitextunivcz}

\begin{definition}[Název definice]
Abcd. Abcd. Abcd. Abcd. Abcd. Abcd. Abcd. Abcd. Abcd. Abcd. Abcd. Abcd. Abcd. Abcd. Abcd. Abcd. Abcd. Abcd. Abcd. Abcd. Abcd. Abcd. Abcd. Abcd. Abcd. Abcd. Abcd. Abcd. Abcd. Abcd. \gls{UPOL}
\end{definition}

\begin{proof}[Název důkazu]
Abcd. Abcd. Abcd. Abcd. Abcd. Abcd. Abcd. Abcd. Abcd. Abcd. Abcd. Abcd. Abcd. Abcd. Abcd. Abcd. Abcd. Abcd. Abcd. Abcd. Abcd. Abcd. Abcd. Abcd. Abcd. Abcd. Abcd. Abcd. Abcd. Abcd.
\end{proof}

\begin{remark}[Pumpovací věta]
Abcd. Abcd. Abcd. Abcd. Abcd. Abcd. Abcd. Abcd. Abcd. Abcd. Abcd. Abcd. Abcd. Abcd. Abcd. Abcd. Abcd. Abcd. Abcd. Abcd. Abcd. Abcd. Abcd. Abcd. Abcd. Abcd. Abcd. Abcd. Abcd. Abcd.
\end{remark}

\begin{example}[Pumpovací věta]
Abcd. Abcd. Abcd. Abcd. Abcd. Abcd. Abcd. Abcd. Abcd. Abcd. Abcd. Abcd. Abcd. Abcd. Abcd. Abcd. Abcd. Abcd. Abcd. Abcd. Abcd. Abcd. Abcd. Abcd. Abcd. Abcd. Abcd. Abcd. Abcd. Abcd.
\end{example}

\begin{lemma}[Název definice]
Abcd. Abcd. Abcd. Abcd. Abcd. Abcd. Abcd. Abcd. Abcd. Abcd. Abcd. Abcd. Abcd. Abcd. Abcd. Abcd. Abcd. Abcd. Abcd. Abcd. Abcd. Abcd. Abcd. Abcd. Abcd. Abcd. Abcd. Abcd. Abcd. Abcd.
\end{lemma}

\begin{consequence}[Název důkazu]
Abcd. Abcd. Abcd. Abcd. Abcd. Abcd. Abcd. Abcd. Abcd. Abcd. Abcd. Abcd. Abcd. Abcd. Abcd. Abcd. Abcd. Abcd. Abcd. Abcd. Abcd. Abcd. Abcd. Abcd. Abcd. Abcd. Abcd. Abcd. Abcd.
\end{consequence}

\begin{theorem}[Pumpovací věta]
Abcd. Abcd. Abcd. Abcd. Abcd. Abcd. Abcd. Abcd. Abcd. Abcd. Abcd. Abcd. Abcd. Abcd. Abcd. Abcd. Abcd. Abcd. Abcd. Abcd. Abcd. Abcd. Abcd. Abcd. Abcd. Abcd. Abcd. Abcd. Abcd. Abcd.
\end{theorem}


\begin{kicode}{cpp}{kod:cpp}{\cpp}
int main("cs acsa") // komentar
int main("cs acsa") // komentar
int main("cs acsa") // komentar
int main("cs acsa") // komentar
int main("cs acsa") // komentar
\end{kicode}

\begin{kicode}{JavaScript}{}{JS}
new object() // komentar
\end{kicode}

\begin{kicode}{csharp}{}{\csharp}
public static int main("cs acsa") // komentar
\end{kicode}

\begin{kicode}{SQL}{}{SQL}
SELECT * FROM table_1; /* komentar */
\end{kicode}

\begin{kicode}{TutorialD}{}{TutorialD}
table_1 AND table_2;
\end{kicode}

%% Závěry práce. V jazyce práce a anglicky. Text pro jiný než
%% nastavený jazyk práce (nepovinným parametrem language makra
%% \documentclass, výchozí český) se zadává použitím makra s uvedením
%% jazyka jako nepovinného parametru.
\begin{kiconclusions}
Závěr práce v \uv{českém} jazyce.
\end{kiconclusions}

\begin{kiconclusions}[english]
Thesis conclusions in \uv{English}.
\end{kiconclusions}

%% Přílohy obsahu textu práce, za makrem \appendix.
\appendix

\section{První příloha}
Text první přílohy

\section{Druhá příloha}
Text druhé přílohy

%% Obsah elektronických dat. Poslední příloha. Upravte podle vlastní
%% práce!
\section{Obsah elektronických dat} \label{sec:ObsahData}

Na samotném konci textu práce je uveden stručný popis obsahu
elektronických dat odevzdaných v systému katedry informatiky spolu s
textem. Tato data jsou nedílnou součástí práce a tvoří (datovou)
přílohu textu práce. Povinné položky struktury dat jsou:

\begin{description}

\item[\texttt{text/}] \hfill \\
  Adresář s textem práce ve formátu PDF, vytvořený s~použitím
  závazného stylu KI PřF UP v~Olomouci pro závěrečné práce, včetně
  všech (textových) příloh, a~všechny soubory potřebné pro
  bezproblémové vytvoření PDF dokumentu textu (případně v~ZIP
  archivu), tj.~zdrojový text textu a příloh, vložené obrázky, apod.

\item[\texttt{README.*}] \hfill \\
  Textový soubor (s příponou např. \texttt{.txt}) s informacemi o
  opakovatelném způsobu použití ostatních dat práce -- typicky plně
  reprodukovatelný co nejúplnější funkční postup zprovoznění software
  vytvořeného v~rámci práce, tzn. jeho případné instalace/nasazení a
  spuštění, včetně uvedení všech požadavků pro bezproblémový provoz;
  za zprovoznění software se nepovažuje zpřístupnění (např. po
  Internetu) již někde zprovozněného software.

\item[\texttt{*}] \hfill \\
  Adresáře a soubory s veškerými ostatními autorskými daty práce
  (případně v~ZIP archivu) -- typicky spustitelné a další soubory
  software vytvořeného v rámci práce potřebné pro bezproblémový provoz
  software, případně jeho instalační program, a kompletní zdrojové
  texty software a další data nutná pro plně reprodukovatelné korektní
  vytvoření spustitelných souborů.

\end{description}

Dále mohou data obsahovat například:

\begin{itemize}

%\item[\texttt{data/}] \hfill \\
\item
  ukázková a~testovací data použitá v~práci nebo pro potřeby posouzení
  práce v rámci její obhajoby,

%\item[\texttt{literature/}] \hfill \\
\item
  položky bibliografie v elektronické podobě, příp.~jiná relevantní literatura
  a dokumentace vztahující se k~práci,

%\item[\texttt{install/}] \hfill \\
\item
  cizí data (software) potřebná pro bezproblémové použití autorských
  dat práce (software), která nejsou standardní součástí
  předpokládaného (softwarového) vybavení uživatele.

\end{itemize}

U~veškerých cizích obsažených materiálů jejich
zahrnutí dovolují podmínky pro jejich veřejné šíření nebo přiložený souhlas
držitele práv k užití. Pro všechny použité (a~citované) materiály,
u~kterých toto není splněno a~nejsou tak obsaženy, je uveden
jejich zdroj, např.~webová adresa, v~bibliografii nebo textu práce
nebo souboru \texttt{README.*}.

%% -------------------------------------------------------------------

%% Sazba volitelného seznamu zkratek, za přílohami.
\printglossary

%% Sazba povinné bibliografie, za přílohami (případně i za seznamem
%% zkratek). Při použití BibLaTeXu použijte makro
%% \printbibliography. jinak prostředí thebibliography. Ne obojí!

%% Sazba i v textu necitovaných zdrojů, při použití
%% BibLaTeXu. Volitelné.
\nocite{*}
%% Vlastní sazba bibliografie při použití BibLaTeXu.
\printbibliography

%% Bibliografie, včetně sazby, při NEpoužití BibLaTeXu.
% \begin{thebibliography}{9}
%\bibitem{kniha2} \uppercase{Hawke}, Paul. NanoHttpd: Light-weight HTTP server designed for embedding in other applications. GitHub [online]. 2014-05-12. [cit. 2014-12-06]. Dostupné z: \url{https://github.com/NanoHttpd/nanohttpd}
%
%\bibitem{jeske13} \uppercase{Jeske}, David; \uppercase{Novák}, Josef. Simple HTTP Server in \csharp: Threaded synchronous HTTP Server abstract class, to respond to HTTP requests. CodeProject: For those who code [online]. 2014-05-24. [cit. 2014-12-06]. Dostupné z: \url{http://www.codeproject.com/Articles/137979/Simple-HTTP-Server-in-C}
%
%\bibitem{uzis2012} \uppercase{ÚSTAV ZDRAVOTNICKÝCH INFORMACÍ A STATISTIKY ČR}. Lékaři, zubní lékaři a farmaceuti 2012 [online]. Praha 2, Palackého náměstí 4: Ústav zdravotnických informací a statistiky ČR, 2012 [cit. 2014-12-06]. ISBN 978-80-7472-089-5. Dostupné z: \url{http://www.uzis.cz/publikace/lekari-zubni-lekari-farmaceuti-2012}
% \end{thebibliography}

%% Sazba volitelného rejstříku, za bibliografií.
\printindex

\end{document}

%%% Local Variables:
%%% mode: latex
%%% TeX-master: t
%%% End:
